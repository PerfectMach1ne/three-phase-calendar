\documentclass[12pt, a4paper]{article}
\usepackage[utf8]{inputenc}
\usepackage[T1]{fontenc}
\usepackage[polish, english]{babel}
\usepackage{enumitem}

\usepackage[
    backend=biber,
    style=numeric,
    sorting=nty
]{biblatex}
\addbibresource{biblio.bib}

\usepackage{tikz}
\usepackage{pdflscape}
\usepackage{array}
\usepackage{booktabs}
\usetikzlibrary{positioning,shapes.geometric}

\title{Modalna aplikacja webowa oraz mobilna wspomagająca organizację czasu}
\author{Luka M[]}
\date{08.2024 - 05.2025}
\begin{document}
\maketitle

\section*{Wprowadzenie}
W dzisiejszym świecie, branża inżynierii oprogramowania zdołała już postawić wiele kroków w stronę rewolucyjnego postępu. Liczna liczba narzędzi w postaci oprogramowania umożliwiła osobom o wyjątkowo róznorodnym doświadczeniu wejść do branży IT oraz inżynierii oprogramowania, w szczególności w specjalizacji technologii webowych.

Bardzo prawdopodobnym bezpośrednim skutkiem tego szybkiego rozwoju jest popularyzacja określonych metodologii tworzenia oprogramowania, frameworków, bibliotek oraz różnych innych rozwiązań. Zjawisko to spowodowało, że wiele rozwiązań programistycznych, zarówno amatorskich, open source-owych i firmowych, dotyczących powszechnie spotykanych problemów dzieli wiele podobieństwd w ich sposobie implementacji oraz wdrożenia.

Głównym celem tego projektu jest zakwestionowanie powszechnych podejść do często spotykanych problemów, przy jednoczesnym utrzymaniu się przy  ugruntowanych standardach oraz sprawdzonych najlepszych praktykach w wytwarzaniu oprogramowania.
\section{Zakres i cele projektu}
\section{Technologie wykorzystane w procesie wytwarzania oprogramowania}
Technologie wykorzystane zarówno na etapie projektowania, implementacji \& wdrożenia projektu składają się z połączenia najlepszych narzędzi sprawdzonych do danego zadania oraz wyboru preferencji dewelopera.
\subsection{Środowisko programistyczne}
[System operacyjny na jakim został zaprogramowany: EndeavourOS Linux \\
Alpine Linux - obraz Dockera! \\
Edytory tekstu oraz IDE - VS Code, Neovim, Vim, etc \\]
\subsection{Java 21 (OpenJDK)}
\subsection{PostgreSQL}
\subsection{Vue.js}
\subsection{Tauri 2.0}
\subsection{Infrastruktura}
Wdrożenie w chmurze - AWS! Docker!
\subsection{Testing tooling}

\section{Projekt oraz proces implementacji}

\subsection{Serwer i mikroframework backendowy}

U fundamentów projektu leży API dostarczane przez pakiet \\\texttt{com.sun.net.httpserver} dostępny w większości dystrybucji Javy, w tym wykorzystywanej przez projekt OpenJDK 21. Pakiet zawiera klasę abstrakcyjną \textbf{HttpServer} oraz jej podklasę \textbf{HttpsServer} \cite{httpsserver}, obydwie dostarczające bardzo proste API serwera HTTP/HTTPS. Obiektowi \texttt{HttpServer} przypisuje się adres IP oraz numer portu, po czym serwer od razu może oczekiwać na połączenia TCP przychodzące na podany adres.
Poza klasą \texttt{HttpServer}, której egzemplarz stanowi pojedyńczą instancję serwera HTTP, głównymi komponentami ów API są następujące klasy \cite{javapkg-httpserver}:
\begin{itemize}
    \item Intefejs \textbf{HttpHandler}, którego egzemplarze przypisuje się obiektowi \texttt{HttpServer} przy tworzeniu obiektu serwera HTTP. Programista musi samodzielnie dostarczyć implementacji metody \\\texttt{void handle(HttpExchange)}, która obsługuje parę żądania i odpowiedzi HTTP;
    \item Klasa \textbf{HttpExchange} dostarczająca abstrakcji i metod do obsługi pary żądania oraz odpowiedzi zgodnej z standardem HTTP;
    \item Klasa \textbf{HttpContext} reprezentująca odwzorowanie adresu URI do poszczególnego egzemplarza \texttt{HttpHandler}; jej obiekty są automatycznie tworzone przy wywołaniu metody \texttt{HttpServer.create(InetSocketAddress addr, int backlog)};  
    \item Klasa \textbf{Filter} umożliwiająca pre- i post-procesowanie żądań konieczne do obsługi parametrów zapytania HTTP, e.g.:  \\
\texttt{http://www.example.com/context?param=value\&?abc=123}
\end{itemize}

Pierwszym problemem napotkanym przy projektowaniu aplikacji była rozbieżność między możliwościami narzędzi dostarczanych przez API \texttt{httpserver} oraz koniecznością uporządkowania kodu w strukturę umożliwiającą podtrzymywalność, debugowalność oraz rozszerzalność serwera.
Wykorzystane rozwiązanie opiera się na pospolitych wzorcach projektowych spotykanych w frameworkach backendowych, w tym rozdzieleniu zadań serwera na podsystemy skupione na konkretnych funkcjonalnościach: utworzenie instancji serwera, obsługa żądań i odpowiedzi HTTP, walidacja i przetwarzanie danych, oraz obsługa zapytań bazy danych. Następnie programiście dostarczane jest drugorzędne API abstrahujące logikę serwera spajającą wszystkie podsystemy, umożliwiając programiście skupienie się na domenie i logice biznesowej wyższego poziomu.

\subsubsection{Struktura aplikacji serwerowej}
[App class \& app structure overview, also utils/],

\subsubsection{Łączność i schemat bazy danych}
[SchemaInitializer \& basic db connection],

\begin{landscape}
\begin{figure}[h]
\centering
\begin{tikzpicture}[
    node distance=1.25cm and 1.25cm,
    table/.style={
        draw,
        rectangle,
        minimum width=3cm,
        inner sep=3pt,
        fill=white
    },
    enum/.style={
        draw,
        trapezium,
        trapezium stretches body,
        fill=yellow!5,
        minimum width=2.5cm
    },
    annotation/.style={
        font=\scriptsize\itshape,
        text width=5cm
    }
]

%%% NODES AS TEXT TABLES %%%
% Users
\node[table] (users) {
    \begin{tabular}{>{\ttfamily}l}
    \multicolumn{1}{c}{\textbf{users}} \\
    \midrule
    PK: id \\
    name: VARCHAR(36) \\
    email: TEXT \\
    password: VARCHAR(36) \\
    \bottomrule
    \end{tabular}
};

% Calendarspace
\node[table, right=of users] (calendarspace) {
    \begin{tabular}{>{\ttfamily}l}
    \multicolumn{1}{c}{\textbf{calendarspace}} \\
    \midrule
    PK: id \\
    FK: user\_id \\
    taskevents\_arr: INT[] \\
    timeblocks\_arr: INT[] \\
    textevents\_arr: INT[] \\
    \bottomrule
    \end{tabular}
};

% Taskevents
\node[table, above left=of calendarspace] (taskevents) {
    \begin{tabular}{>{\ttfamily}l}
    \multicolumn{1}{c}{\textbf{taskevents}} \\
    \midrule
    PK: id \\
    hashcode: INT \\
    datetime: TIMESTAMP \\
    name: TEXT \\
    description: TEXT \\
    isdone: BOOLEAN \\
    \bottomrule
    \end{tabular}
};

% Timeblocks
\node[table, above right=of calendarspace] (timeblocks) {
    \begin{tabular}{>{\ttfamily}l}
    \multicolumn{1}{c}{\textbf{timeblockevents}} \\
    \midrule
    PK: id \\
    hashcode: INT \\
    start\_datetime: TIMESTAMP \\
    end\_datetime: TIMESTAMP \\
    name: TEXT \\
    description: TEXT \\
    \bottomrule
    \end{tabular}
};

% Textevents
\node[table, below right=of timeblocks] (textevents) {
    \begin{tabular}{>{\ttfamily}l}
    \multicolumn{1}{c}{\textbf{textevents}} \\
    \midrule
    PK: id \\
    hashcode: INT \\
    content: TEXT \\
    \bottomrule
    \end{tabular}
};

% Viewtype enum
\node[enum, above=of taskevents] (viewtype) {
    \begin{tabular}{c}
    \textbf{viewtype} \\
    \midrule
    static\_task \\
    historic\_task \\
    routine\_task \\
    \bottomrule
    \end{tabular}
};

%%% ANNOTATIONS %%%
\node[annotation, below=of calendarspace] (ann-cal) {
    \begin{minipage}{3cm}
    \footnotesize\textit{Przykład:} \\
    \texttt{\{"user\_id": 1,} \\
    \texttt{"taskevents\_arr": [102,107]\}}
    \end{minipage}
};

%%% CONNECTIONS %%%
\draw[->, thick] (users) -- (calendarspace);
\draw[->, thick] (calendarspace) |- node[pos=0.3, left] {1:N} (taskevents);
\draw[->, thick] (calendarspace) -- ++(0.8,1.75) |- node[pos=0.7, below] {1:N} (timeblocks);
\draw[->, thick] (calendarspace) -- ++(3,0) |- node[pos=0.7, below] {1:N} (textevents);
\draw[->, dashed] (taskevents) -- (viewtype);
\draw[->, dashed] (timeblocks) -- ++(0,2) |- (viewtype);
\draw[->, dashed] (textevents) -- (textevents |- {0,9}) -- (viewtype);

\end{tikzpicture}
\caption{Diagram ER schematu bazy danych}
\label{fig:hybrid-er}
\end{figure}
\end{landscape}

\subsubsection{Abstrakcja obsługi URI i metod HTTP}
[APIContexts, HTTPContext wrapper, QueryParamsFilter; HTTP additions],

\subsubsection{Routery HTTP}
[APIRouter \& subclasses; HttpExchange indepth],

\subsubsection{Obsługa operacji CRUD bazy danych}
[DBProxy classes],

\subsubsection{Walidacja danych}
[JSON schema validation],

\subsubsection{Mechanizmy bezpieczeństwa}
[security, RoleInjector],

\subsection{Frontend}
[UI ZONE
MVC i MVVM z ksiązki o Vue 2]


[JavaScript dates have been giving me growing pains due to ograniczone możliwości obiektu Date z ES6 w przeciwieństwie do Javy]

\section{Integracja, konserwacja oraz szerszy obraz}
\section*{Podsumowanie}

\medskip

\printbibliography[title={Bibliografia}]

\subsection*{Index of pictures}

\end{document}